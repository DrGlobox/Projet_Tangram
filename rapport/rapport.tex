\documentclass[a4paper, 11pt]{report}



\input{modele}

	%\title{\vspace{5cm}Nascar \\ Rapport de projet \\ LO41 \\ \ \\}
	%\date{automne 2012\\ \ \\ \vspace*{3cm}
	%\includegraphics[width=4cm]{logo_utbm.png}
	%}
	%\author{Paul Locatelli - Pierre Rognon \\ \ \\ \ \\ Université de Technologies de Belfort-Montbéliard\\ \ \\}
	
	\titleformat{\chapter}[hang]{\bf\huge}{\thechapter}{2pc}{}     

\begin{document}

\makeatletter
	\def\clap#1{\hbox to 0pt{\hss #1\hss}}%
	\def\ligne#1{%
	\hbox to \hsize{%
	\vbox{\centering #1}}}%
	\def\haut#1#2#3{%
	\hbox to \hsize{%
	\rlap{\vtop{\raggedright #1}}%
	\hss
	\clap{\vtop{\centering #2}}%
	\hss
	\llap{\vtop{\raggedleft #3}}}}%
	\def\bas#1#2#3{%
	\hbox to \hsize{%
	\rlap{\vbox{\raggedright #1}}%
	\hss
	\clap{\vbox{\centering #2}}%
	\hss
	\llap{\vbox{\raggedleft #3}}}}%
	\def\maketitle{%
	\thispagestyle{empty}\vbox to \vsize{%
	\haut{}{\@blurb}{}
	\vfill
	\vspace{1cm}
\begin{flushleft}
	%\usefont{OT1}{ptm}{m}{n}
	\huge \@title
\end{flushleft}
	\par
	\hrule height 4pt
	\par
\begin{flushright}
	%\usefont{OT1}{phv}{m}{n}
	\Large \@author
	\par
\end{flushright}
	\vspace{1cm}
	\vfill
	\vfill

\begin{center}
	\includegraphics[width=5cm]{logo_UTBM.jpg}
\end{center}

\bas{}{Automne 2012}{}
}%
\cleardoublepage
}
\def\date#1{\def\@date{#1}}
\def\author#1{\def\@author{#1}}
\def\title#1{\def\@title{#1}}
\def\location#1{\def\@location{#1}}
\def\blurb#1{\def\@blurb{#1}}
\date{\today}
\author{}
\title{}

% informations
\location{Belfort}\blurb{}
\makeatother
\title{Projet de résolution d'un tangram}
\author{\small{Adrien \bsc{Berthet} Paul \bsc{Locatelli} et Pierre \bsc{Rognon}}}
\blurb{%
	\textbf{IA41 - Concepts fondamentaux en Intelligence Artificielle et langages dédiés}\\
	Université de Technologie de Belfort-Montbéliard
}% 


	
	\maketitle
	
	\newpage
	
	\shorttoc{Sommaire}{0}
	
	
	\chapter*{Introduction}
	\addcontentsline{toc}{chapter}{Introduction}
	
		Pour ce projet, le sujet abordant le casse-tête du tangram (numéro 6) a été retenu. Le choix de ce sujet s'est fait sur l'intérêt de ce jeu. En effet, s'il apparaît assez aisé de résoudre des modèles tels que le classique "carré" que chacun connaît, d'autres modèles sont beaucoup plus complexes. De plus, c'est un test utilisé dans de nombreux tests afin de déceler d'éventuelles déficiences chez les enfants tels que le WISC (Wechsler Intelligence Scale for Children). C'est donc un problème d'Intelligence Artificielle fondamental qui sera abordé dans ce projet et dont la résolution sera tentée.\\
L'analyse de ce problème constitue une partie prépondérante dans la tentative de résolution du tangram. Environ 6000 configurations différentes sont connues aujourd'hui. De nombreux modèles ont donc dû être envisagés afin de couvrir l'ensemble des cas possibles. Une seconde phase qui s'avère aussi complexe est le choix des différentes représentations informatiques du problème. De nombreux traitements ont été effectués pour mettre à bien cette partie. Quelques exemples qui ont été utilisés comme support durant toute la durée du projet seront présentés afin d'indiquer leur solution. Enfin, la résolution par le logiciel crée sera abordée pour ces exemples. Les limites de ce problème et l'état d'avancement du projet seront indiqués enfin.


	\newpage

	\chapter{Analyse du casse-t\^ete}
	
	Une première analyse du tangram a permis de mettre à jour plusieurs sous-problèmes. Deux problèmes d'ordre mathématique ainsi qu'un problème d'intelligence artificielle pure ont ainsi été isolés. 
	
	
	\newpage
	
	\chapter{Représentation informatique du problème}
	
	Le passage de la phase d'analyse à la phase "informatique" du pour la résolution du tangram s'avère beaucoup plus complexe que ce qui avait été prévu. Cependant, des représentations précises ont été adoptées.
	
	\chapter{Résultats obtenus}
	
	De nombreux problèmes ayant été rencontrés durant le projet, les résultats obtenus n'indiquent pas une résolution totale de différents modèles. Ceci est d\^u au fait des différentes difficultés apparues sur les algorithmes mathématiques touchant à la géométrie.
	
	
	
	
	
	


		
	\newpage	
		
	\tableofcontents
		
\end{document}