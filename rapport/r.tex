\documentclass[a4paper, 11pt]{report}
\usepackage{appendix}

\begin{document}
\begin{appendices}

    \chapter{Prédicat soustraction}
    \begin{verbatim}
    % file : soustraction.pl
    soustraction([Patterns], Piece, [NewPatterns]):-

    tools:liste_all_couple_aretes(Patterns,CoupleAretePatterns),
    tools:liste_all_couple_aretes(Piece,CoupleAretePiece),

    search_commmon_arete(CoupleAretePatterns,CoupleAretePiece,Arete),
    search_sens_forme(CoupleAretePiece,Arete,Sens),
    retourne_forme(CoupleAretePiece,Sens,CoupleAretePiece1),

    mix_pattern_forme(CoupleAretePatterns,CoupleAretePiece1,Arete,Mixed),
    clean_double_arete(Mixed,CleanedMixed),

    arete_to_point(CleanedMixed,NewListPoint),
    clean_double_point(NewListPoint,NewPatterns1),
    clean_first_last_double_point(NewPatterns1,NewPatterns).
    \end{verbatim}

    \begin{itemize}
        \item \textbf{Patterns} : variable représentant le modèle actuel à réaliser;
        \item \textbf{Piece} : variable représentant la pièce positionné en fonction du modèle;
        \item \textbf{NewPatterns} : variable représentant le nouveau modèle.
    \end{itemize}

\chapter{Prédicat placePiece}
    \begin{verbatim}

% file : placePiece.pl
placePiece(Piece,Patterns,ReturnPiece):-
    test_placement_patterns_exact_match(Piece,Patterns,Placement),
    positionne(Piece,Placement,NewPiece),
    test_all_placement(NewPiece,Patterns,Placement,ReturnPiece).

placePiece(Piece,Patterns,ReturnPiece):-
    test_placement_patterns_angle_match(Piece,Patterns,Placement),
    positionne(Piece,Placement,[Axe1,Axe2],NewPiece),
    test_placement(NewPiece,Patterns,[Axe1,Axe2],ReturnPiece).

    \end{verbatim}

    \begin{itemize}
        \item \textbf{Piece} : variable représentant la pièce à placer;
        \item \textbf{Patterns} : variable représentant le modèle où placer la pièce;
        \item \textbf{ReturnPiece} : variable représentant la pièce correctement placée.
    \end{itemize}


\chapter{Prédicat essai_piece}
    \begin{verbatim}

% file : main.pl
essai_piece(_,[[]],_,_):-!.
essai_piece([], _, _, _) :- !.
essai_piece(_, [], Solution, _) :- !, 
    affiche_resultat(Solution).

essai_piece(_, _, _, 0) :- 
    write_ln('\tBout de la branche atteint'), !, fail.

essai_piece([Piece|PiecesRestantes], Pattern, [[Piece, Placements]|PiecesRetenues], _) :-
    points_piece(Piece, PointsPiece),
    write('Essai de la pièce '),
    write(Piece),
    placePiece(PointsPiece, Pattern, Placements),
    retirer_piece(PiecesRestantes, Pattern, Placements, PiecesRetenues).

essai_piece([PieceNonUtilisee|PiecesRestantes], Pattern, PiecesRetenues, N) :- 
    append(PiecesRestantes, [PieceNonUtilisee], Pieces), N1 is N-1,
    write_ln(' -> la pièce est mise de côté pour le moment'),
    essai_piece(Pieces, Pattern, PiecesRetenues, N1).  
    
    \end{verbatim}

    \begin{itemize}
        \item \textbf{ListePiece} : variable représentant la liste des pièces à placer;
        \item \textbf{Pattern} : variable représentant le modèle à réaliser;
        \item \textbf{PiecesRetenues} : variable représentant la liste des pièces placées;
        \item \textbf{N} : variable représentant la profondeur maximale de l'arbre de réalisation.
    \end{itemize}




\end{appendices} 
\end{document}

